%   \vspace{-1mm} 
  High-performance computing systems using accelerators or co-processors have become popular, of which Nvidia GPU and 
Intel Xeon Phi Many Integrated Cores (MIC) are two main architectures.  
A general purpose GPU includes 100x to 1000x simpler cores (compared against general-purpose CPU cores) 
and realizes single instruction multiple thread execution model 
that excels in processing data parallel workloads in which inter-core
communication are minimum. 
MIC employs a hybrid architecture combining classic x86 based many-core architecture (\verb#~#70 cores) 
and enhanced vector processing units, thus able to deliver performance improvement for applications with either task parallelism or data parallel, or
the mix of both than conventional CPUs. The power envolopes of these accelerators and CPUs are also different in both idle and loaded states. 

The architectures and programming models of MIC are similar to CPUs,
allowing for easier migration of legacy applications that were developed for multicore CPUs
to use the hardware processing capability. Programming languages such as C/C++/FORTRAN with OpenMP/MPI~\cite{R:6,R:20} are available for programming
on CPU and MICs. 
GPUs employ a totally different architecture and execution model than CPU/MIC, as well as new programming model such as CUDA~\cite{R:21}. Though 
existing programming models, such as OpenMP and OpenACC have been enhanced with goals to provide 
a unified model for programming CPU/GPU/MIC, performance
computing for real-world application on those architectures still rely on vendor-specific solutions, especially before robust implementations of 
those models are available for supporting all the three architectures. 

%have . 
%Both MIC and CPU are however highly different from GPU. 
 
  % High performance computing using accelerators or co-processors has been popular. In addition to CPU, Nvidia GPU and Intel Xeon Phi Many Integrated Cores (MIC) have gradually become two main architectures in the market of high performance and scientific applications, which requires high throughput and intensive computing power. 
   


The hardware architecture heterogeneity and software variants of programming models pose a difficult problem for users to 
choose suitable accelerators or coprocessors for specific applications considering both performance and usability of each architecture. 
Comprehensively understanding the characteristics of heterogeneous architectures in various aspects can help us in 
designing more efficient applications, choosing the appropriate accelerators for an application, 
and developing more effective task scheduling and mapping strategies. In this paper, we investigate and compare the performance, 
efficiency, power consumption, energy efficiency, temperature, productivity and cost of the three architectures using representative 
benchmarks of Rodinia~\cite{R:1}. 
Our work differs from other related work that concentrates on the comparison of performance and/or energy 
consumption of two or three of architectures on specific applications~\cite{halyo2014first, lee2010debunking, R:10, R:12, R:8, R:9} by providing a comprehensive analysis of those metrics. We thus
providing more insightful guidance for users in selecting accelerators for applications.  
% within different domains on CPU, GPU and MIC architectures. 
This paper makes the following contributions: 


%The development of heterogeneous architectures promotes researchers to develop a lot of benchmark suites to evaluate the emerging class of architectures and to find out the strength and weakness between heterogeneous architectures. The most popular open source benchmark suites targeting heterogeneous architectures include Rodinia, Parboil \cite{R:2} and the Scalable Heterogeneous Computing (SHOC) \cite{R:3}. These benchmark suites provide applications implemented in different programming models for heterogeneous architectures and fulfill roles similar to PARSEC \cite{R:4} and SPEC \cite{R:5} that are other popular benchmark suites managed by an industry consortium. These benchmarks also help architects study programming models and applications concurrently. 

%The Rodinia applications are designed for heterogeneous computing infrastructures using OpenMP, CUDA, and OpenCL \cite{R:7} targeting CPUs and GPUs to evaluate these two architectures separately. It provides representative real world applications from multiple domains such as Data Mining, Linear Algebra, Fluid Dynamics, Graph Algorithms, Physics Simulation ,Bioinformatics, etc. We carry out a set of comprehensive performance analysis by using the OpenMP and CUDA programming models for these applications.  Although the performance analysis is our main motivating purpose, our metric and approaches for comprehensive performance analysis can be used for other heterogeneous benchmark such as Parboil and SHOC. 
  

  
\begin{itemize}    
\item  We present a comprehensive performance analysis including performance, power consumption, temperature, productivity and monetary cost for five representative benchmarks of Rodinia on CPU, Nvidia GPU and Intel MIC architectures.    
\item  We evaluate the performance efficiency with regard to practical and theoretical performance, energy efficiency with regard to performance and power consumption, and monetary cost efficiency with regard to performance and monetary cost of the three architectures.    
\item  We propose a unified efficiency metric to compare the overall benefits of the three architectures.
\end{itemize}

The rest of the paper is organized as follows: Section 2 reviews the related work. Section 3 provides the short overview of benchmarks of Rodinia. Section 4 presents the comprehensive performance analysis of the benchmarks for CPU, GPU, and MIC. Section 5 concludes the paper.
