It is well known that as the computing technology advances, the size of compute intensive problems also increases so that GPU and MIC, as the most powerful accelerators, have become the latest demanding market for HPC computing compared with CPU. Thus comprehensive performance analysis of heterogeneous architectures has become an important role. In the report made by You et al. \cite{R:8}, they gave out a new method to evaluation, which is performance efficiency. The formula of it is obtained performance/theoretical performance. The performance efficiency gap between two devices is small in the report. The performance of operations in an important class of applications in the fields of microscopy image analysis on GPU, MIC and CPU architectures are investigated and characterized by George Teodoro et al. \cite{R:9}. They systematically implement and evaluated the performance of operations on modern CPU, GPU, and MIC systems using approaches that exhibit different data access patterns (regular and irregular), computation intensity, and types of parallelism, from a class of applications. 

For the cluster systems, Massimo Bernaschi et al. \cite{R:10} presented and compared the performances of GPU and MIC both in a single system and in cluster configuration for the simulation of two physical systems showing that the performances of an MIC change dramatically depending on the need to pad arrays to avoid dramatic performance drops due to L2 TLB trashing effects. In this paper \cite{R:11}, authors perform a rigorous performance analysis and find that after applying optimizations appropriate for both CPUs and GPUs, the performance gap between an Nvidia GTX280 processor and the Intel Core i7 960 processor narrows to only 2.5x on average. They also discuss optimization techniques for both CPU and GPU, analyze what architecture features contributed to performance differences between the two architectures, and recommend a set of architectural features which provide significant improvement in architectural efficiency for throughput kernels. Performance and evaluation comparison of Multi Text Keyword Search algorithms into MIC and GPU has also been proposed and evaluated by performance analysis \cite{R:12}. Authors used Nvidia K20c and Nvidia K40 for their GPUs and Intel Xeon Phi 5100 for MIC, and found out that K20c and K40 outperformed MIC for this particular algorithm. 

Researchers generally compare heterogeneous architectures by performance of applications instead of other critical factors such as performance efficiency, energy efficiency, monetary cost, temperature, productivity, etc. Some performance analysis of heterogeneous architectures are only based on the applications they developed instead of well known benchmarks that can be a standard comparison to fairly evaluate the heterogeneous architectures. In addition, researchers compare GPU with CPU while MIC has played an important role in high performance computing, or compare GPU with MIC while CPU has strength for some applications. Thus the importance of comprehensive comparison between GPU, MIC and CPU is obvious. Analytical models of CPUs \cite{R:13} and GPUs \cite{R:14} have also been proposed. They provide a structural understanding of throughput computing performance, which only focus on a homogeneous architecture for CPU or GPU instead of heterogeneous architectures. Base on the reasons above, we propose comprehensive performance analysis and a unified efficiency metric to evaluate heterogeneous architectures.  

